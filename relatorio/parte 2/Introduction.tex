\paragraph{} 
Nesta segunda parte do trabalho pretende-se implementar um processador com o mesmo tipo de arquitectura anterior mas com um funcionamento em pipeline. A introdução da execução em pipeline traz alguns problemas devido à existência de dependências dados e de controlo entre as várias instruções. Foram utilizadas algumas técnicas para resolver os conflitos gerados por estas dependências da forma mais eficiente possível, isto é, com utilização do menor número de ciclos adicionais para resolver os conflitos e com a frequência máxima possível. Após a implementação destas técnicas é feita uma comparação das mesmas para diferentes situações práticas.
